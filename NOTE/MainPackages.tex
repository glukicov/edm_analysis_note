%!TEX root = PhDThesis_Gleb_Lukicov.tex
% -------- Packages --------
\usepackage[utf8]{inputenc}

%%%%%%%%% Fonts and topgraphy
\usepackage[english]{babel} %topography
\usepackage{xspace} % /xspace command in macros 
%%%%%%%%%

%%%%%%%%%%%%% Not used but useful packages 
% \usepackage{feynmf} % Feynamn diagrams  
% \usepackage{xcolor} % \textcolor{colour}{text}
% \usepackage{isorot} % The isorot package allows you to put things sideways (or indeed, at any angle) on a page.
% \usepackage{etoolbox} % If you're getting into defining your own commands, you might want 
% \usepackage[T1]{fontenc}  % set font here 
% \usepackage{tgbonum}  % set font here 
% \addto\captionsenglish{\renewcommand{\figurename}{Fig.}}
% \addto\captionsenglish{\renewcommand{\tablename}{Tab.}}  
%%%%%%%%%%%%%%%%%%%%%%%%%%%%%%%%%%%%%%% 

%%%%%%%%%%%%%%%%%%%%%% General %%%%%%%%%%%%%%%%%%%%%%%%%%%%%%%%%%%%%%%%%%%%%%%%%%%%%% 
% set depth of sub-sub..sections
\setcounter{secnumdepth}{3}  % In the document itself  
\setcounter{tocdepth}{1}   % in the ToC 

\usepackage{setspace} %The setspace package lets you use 1.5-sized or double line spacing.
\setstretch{1.5}

%%%%
% Sharing online watermark
% \usepackage{draftwatermark}
% \SetWatermarkText{\shortstack{\textsc{Copyright: Gleb Lukicov}\\ \strut \\ \strut \\ \strut}}
% \SetWatermarkScale{0.5}
% \SetWatermarkAngle{90}
%%%

\usepackage{ifdraft}
\ifdraft{
 % \usepackage{refcheck} % check labelled but unused floats, tables, etc. 
 \usepackage{draftwatermark}
 \SetWatermarkText{\shortstack{\textsc{Draft Mode}\\ \strut \\ \strut \\ \strut}}
 \SetWatermarkScale{0.5}
 \SetWatermarkAngle{90}
}
%%%%%%%%%%%%%%%%%%%%%%%%%%%%%%%%%%%%%%%%%%%%%%%%%%%%%%%%%%%%%%%%%%%%%%%%%%%%%

%%%%%%%%%%%%%%%%%%%%%%%%%%% Bibliography, HyperRef, and Gloassary  %%%%%%%%%%%%%%%%%%%%%%%%%%%%%%%%%%%%%%%%%%%%%%%%%%%%%%%%%%%%%%%%%%
% Some hacks are necessary to make bibentry and hyperref play nicely.
% See: http://tex.stackexchange.com/questions/65348/clash-between-bibentry-and-hyperref-with-bibstyle-elsart-harv
\usepackage{bibentry}
\makeatletter\let\saved@bibitem\@bibitem\makeatother
\bibliographystyle{phreport} % change to phaip for no titles in articles 
% more styles are here (e.g /usr/local/texlive/2018/texmf-dist/bibtex/bst/beebe)
\renewcommand\bf{\bfseries}  % fixes bbl bf warning 
% smaller vertical spacing between references in bibliography
\let\OLDthebibliography\thebibliography
\renewcommand\thebibliography[1]{
  \OLDthebibliography{#1}
  \setlength{\parskip}{0pt}
  \setlength{\itemsep}{3pt plus 0.3ex}
}

\usepackage[pdftex,hidelinks]{hyperref} % hidelinks=removes the annoying red boxed 
\usepackage{bookmark}  % place extra bookmarks to be seen in the PDF \pdfbookmark[<level>]{<title>}{<dest>}
\usepackage[printonlyused, withpage]{acronym} % [nolist, withpage, nohyperlinks] are good options 
%%%%%%%%%%%%%%%%%%%%%%%%%%%%%%%%%%%%%%%%%%%%%%%%%%%%%%%%%%%%%%%%%%%%%%%%%%%%%%%%%%%%%%%%%%%%%%%%%%%%%%%%%%%%%%%

%%%%%%%%%%%%%%%%%%%%%%%%%%%%%%%%%%%%%%%%%%% Math %%%%%%%%%%%%%%
\usepackage{amsmath} %  maths, 
\usepackage{subdepth} % forces all subscripts at same hight
\usepackage{siunitx} % \SI{100}{\micro\meter} = 100 um 
\usepackage{mathtools} % extension for amsmath 
\usepackage{amsfonts} % extra symbols 
\usepackage{amssymb}  % extra symbols 
\usepackage{gensymb} %  define symbols
\usepackage{textcomp} % define symbols
%%%%%%%%%%%%%%%%%%%%%%%%%%%%%%%%%%%%%%%%%%%%%%%%%%%%%%%%%%%%%%%%

%%%%%%%%%%% References, Graphs, Tables %%%%%%%%%%%%%%%%%%%%%%%%%%%%%%%%%%%%%%%%%%%%%%%%%%%%%%%%%%%%%%%%%%
\usepackage[capitalise]{cleveref} % use full capitals for references 
\Crefname{section}{Section}{Sections}
\Crefname{chapter}{Chapter}{Chapters}
\Crefname{equation}{Equation}{Equations}
\Crefname{figure}{Figure}{Figures}
\Crefname{tabular}{Table}{Tables}
\usepackage{apptools}
\crefname{subappendix}{\IfAppendix{Section}{Appendix}}{\IfAppendix{Sections}{Appendices}s} % (!) the "s" is intentional
\usepackage{notoccite} % removes citing counting from LoF etc.

% \creflabelformat{equation}{#2#1#3} % if removing () around equation reference if preferred
\usepackage{graphicx} % \includegraphics command,
\usepackage{float} %  HERE, with the [H] option to the float environment.
\usepackage[format=hang,font=small,labelfont=bf]{caption} % caption formatting.
\usepackage{sidecap} % allows vertical captions 
\usepackage{subfig} % \subfloat  
\usepackage{array} % column formatting of tables 
\usepackage{multirow} % The multirow package adds the option to make cells span rows in tables.
\usepackage{booktabs}% http://ctan.org/pkg/booktabs nice tables 
\usepackage{tabularx} % advanced tables 
\newcolumntype{L}{>{\raggedright\arraybackslash}X}

% These settings are from: http://mintaka.sdsu.edu/GF/bibliog/latex/floats.html
% See p.105 of "TeX Unbound" for suggested values. See pp. 199-200 of Lamport's "LaTeX" book for details.
\renewcommand{\topfraction}{0.9}    % max fraction of floats at top
\renewcommand{\bottomfraction}{0.8} % max fraction of floats at bottom
%   Parameters for TEXT pages (not float pages):
\setcounter{topnumber}{2}
\setcounter{bottomnumber}{2}
\setcounter{totalnumber}{4}     % 2 may work better
\setcounter{dbltopnumber}{2}    % for 2-column pages
\renewcommand{\dbltopfraction}{0.9} % fit big float above 2-col. text
\renewcommand{\textfraction}{0.07}  % allow minimal text w. figs
% Parameters for FLOAT pages (not text pages):
\renewcommand{\floatpagefraction}{0.7}  % require fuller float pages
% N.B.: floatpagefraction MUST be less than topfraction !!
\renewcommand{\dblfloatpagefraction}{0.7}   % require fuller float pages
%%%%%%%%%%%%%%%%%%%%%%%%%%%%%%%%%%%%%%%%%%%%%%%%%%%%%%%%%%%%%%%%%%%%%%%%%%%%%%%%%%%%%%%%%%%%%%%%%%%%%%%%%%


%%%%%%%%%%%%%%%%%% Visual/Fancy %%%%%%%%%%%%%%%%%%%%%%%%%%%%%%%%%%%%%%
\usepackage{fancyhdr} % section title and page numbering at the top 
\pagestyle{fancy}
% \fancyhead[CO,C]{\normalfont \rmfamily \rightmark \slshape \hfill \thepage}
\fancyhead[LO]{\nouppercase{\rightmark}}
\fancyhead[RO]{\thepage}
\fancyfoot{}
\setlength\headheight{15.0pt}

%Bold math in titles etc. 
\makeatletter
\g@addto@macro\bfseries\boldmath 
\makeatother
%%%%%%%%%%%%%%%%%%%%%%%%%%%%%%%%%%%%%%%%%%%%%%%%%%%%%%%%%%%%%%%%%%%%%%%

%%%%%%%%%%%%%%%%%%%%%%%%%Ease-of-life%%%%%%%%%%%%%%%%%%%%%%%%%%%%
\usepackage{dirtytalk} % \say{} = " " 
\usepackage{nicefrac} % \nicefrac{1}{2} = 1/2 "in-line"
%%%%%%%%%%%%%%%%%%%%%%%%%%%%%%%%%%%%%%%%%%%%%%%%%%%%%%%%%%%%%%%%%%

%%%%%%%%%%   Code snippet insertion %%%%%%%%%%%%%%%%%%%%%%%%%%%%%%%%%%%%%%       
\usepackage{listings}
\usepackage{color}
\definecolor{codegreen}{rgb}{0,0.6,0}
\definecolor{codegray}{rgb}{0.5,0.5,0.5}
\definecolor{codepurple}{rgb}{0.58,0,0.82}
\definecolor{backcolour}{rgb}{0.95,0.95,0.92}
\lstdefinestyle{mystyle}{
    backgroundcolor=\color{backcolour},   
    commentstyle=\color{codegreen},
    keywordstyle=\color{magenta},
    numberstyle=\tiny\color{codegray},
    stringstyle=\color{codepurple},
    basicstyle=\footnotesize,
    breakatwhitespace=false,         
    breaklines=true,                 
    captionpos=b,                    
    keepspaces=true,                 
    numbers=left,                    
    numbersep=5pt,                  
    showspaces=false,                
    showstringspaces=false,
    showtabs=false,                  
    tabsize=2
}
\lstset{style=mystyle}
%%%%%%%%%%%%%%%%%%%%%%%%%%%%%%%%%%%%%%%%%%%%%%%%%%%%%%%%%%%%%%%%%%%%%%%%%%%%%%%